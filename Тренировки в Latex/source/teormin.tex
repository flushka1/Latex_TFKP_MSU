\section{Теоретический минимум}

\subsection{Запишите неравенства треугольника для комплексных чисел}
$|z_1+z_2|\leqslant|z_1|+|z_2|$ или $|z_1-z_2|\geqslant|z_1|-|z_2|$

\subsection{Дайте определение функции однолистной на некотором множестве}
Однолистной в некоторой области называется такая функция $w=f(z)$, что любым двум различным значениям $z_1 \neq z_2$ из этой области отвечают различные значения функции.

\subsection{Дайте определение показательной функции $e^z$}

\subsection{Дайте определения тригонометрических функций $\sin{z}$, $\cos{z}$, $\tg{z}$, $\ctg{z}$}

\subsection{Дайте определения гиперболических функций $\sh{z}$, $\ch{z}$, $\th{z}$, $\cth{z}$}

\subsection{Дайте определение логарифмической функции $Ln z$}

\subsection{Дайте определение общей степенной функции $z^a (a\neq0)$ }

\subsection{Запишите формулу вычисления интеграла от непрерывной функции комплексной переменной вдоль кусочно гладкой кривой через определённый интеграл}

\subsection{Запишите неравенство для модуля интеграла}

\subsection{Дайте определение функции комплексной переменной, дифференцируемой в точке. Приведите пример}

\subsection{Сформулируйте необходимое и достаточное условие дифференцируемости функции в точке}

\subsection{Дайте определение функции, аналитической в области, не содержащей точку $z=\infty$. Приведите пример}

\subsection{Дайте определение функции, аналитической в точке $z_0 \neq \infty$. Приведите пример}

\subsection{Дайте определение функции, аналитической в точке $z = \infty$. Приведите пример}

\subsection{Запишите условия Коши–Римана в случае, когда $z = x+iy$, $w = f(z) = u(x, y)+iv(x, y)$}

\subsection{Запишите условия Коши–Римана в случае, когда $z = re^{i\phi}$, $w = f(z) = U(r, \phi)+iV (r, \phi)$}

\subsection{Запишите условия Коши–Римана в случае, когда $z = x + iy$, $w = f(z) = \rho(x, y)e^{i\Phi(x,y}$}

\subsection{Запишите условия Коши–Римана в случае, когда $z = re^{i\phi}$, $w = f(z) = R(r,\phi)e^{i\Phi(r,\phi)}$}

\subsection{Дайте определение отображения $w = f(z)$, конформного в точке $z_0 \neq \infty$}

\subsection{Дайте определение функции, гармонической в некоторой области}

\subsection{Дайте определение сопряжённых гармонических в некоторой области функций}

\subsection{Сформулируйте теорему Коши для односвязной области}

\subsection{Сформулируйте теорему Коши для ограниченной области}

\subsection{Сформулируйте теорему Коши для многосвязной области}

\subsection{Сформулируйте теорему об аналитичности интеграла с переменным верхним пределом}

\subsection{Запишите интегральную формулу Коши для односвязной области}

\subsection{Дайте определение интеграла типа Коши}

\subsection{Сформулируйте теорему о среднем}

\subsection{Сформулируйте теорему Лиувилля}

\subsection{Сформулируйте принцип максимума модуля аналитической функции}

\subsection{Сформулируйте принцип минимума модуля аналитической функции}

\subsection{Сформулируйте теорему о почленном интегрировании функционального ряда}

\subsection{Сформулируйте вторую теорему Вейерштрасса о функциональных рядах}

\subsection{Сформулируйте теорему Абеля о степенных рядах}

\subsection{Сформулируйте теорему Коши–Адамара о степенных рядах}

\subsection{Запишите формулу Коши–Адамара для радиуса круга сходимости степенного ряда}

\subsection{Сформулируйте теорему об аналитичности суммы степенного ряда}

\subsection{Сформулируйте теорему Тейлора}

\subsection{Запишите дифференциальную и интегральную формулы для коэффициентов
	разложения аналитической функции в степенной ряд}

\subsection{Запишите формулу для производной порядка n аналитической функции}

\subsection{Сформулируйте теорему Морера}

\subsection{Сформулируйте первую теорему Вейрштрасса о функциональных рядах}

\subsection{Дайте определение нуля z0 ̸= ∞ порядка m аналитической функции}

\subsection{Сформулируйте теорему о нулях аналитической функции}

\subsection{Сформулируйте теорему единственности аналитической функции}

\subsection{Дайте определение аналитического продолжения функции f(z), заданной
	первоначально на некотором множестве E}

\subsection{Сформулируйте принцип аналитического продолжения}
